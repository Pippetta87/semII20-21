\documentclass[10pt,xcolor={usenames},fleqn,mathserif,serif]{beamer}
%%%Usefull link
%tikz-equations:
%http://www.wekaleamstudios.co.uk/posts/creating-a-presentation-with-latex-beamer-equations-and-tikz/
\hypersetup{pdfpagemode=FullScreen}
%% colors
\definecolor{bittersweet}{rgb}{1.0, 0.44, 0.37}
\definecolor{brilliantlavender}{rgb}{0.96, 0.73, 1.0}
\definecolor{antiquefuchsia}{rgb}{0.57, 0.36, 0.51}
\definecolor{violetw}{rgb}{0.93, 0.51, 0.93}
\definecolor{Veronica}{rgb}{0.63, 0.36, 0.94}
\definecolor{atomictangerine}{rgb}{1.0, 0.6, 0.4}
\definecolor{darkgray}{rgb}{0.66, 0.66, 0.66}
\definecolor{brightcerulean}{rgb}{0.11, 0.67, 0.84}
\definecolor{cadmiumorange}{rgb}{0.93, 0.53, 0.18}
\definecolor{ochre}{rgb}{0.8, 0.47, 0.13}
\definecolor{midnightblue}{rgb}{0.1, 0.1, 0.44}
\definecolor{lemon}{rgb}{1.0, 0.97, 0.0}
\definecolor{grey}{rgb}{0.7, 0.75, 0.71}
\definecolor{amber}{rgb}{1.0, 0.75, 0.0}
\definecolor{almond}{rgb}{0.94, 0.87, 0.8}
\definecolor{bf}{RGB}{88, 86, 88}
\definecolor{bb}{RGB}{177, 177, 177}
%%%%%%%%%%%%%%%%%%%%%%%%%%%%%%%%%%% importa pacchetti
\usepackage{usepkg}
%%%%%%%%%%%%%%%%%%%%%%%%%%%%%%%%%%% Funzioni generali
\usepackage{functions}
%http://tex.stackexchange.com/questions/246/when-should-i-use-input-vs-include
\newcommand{\setmuskip}[2]{#1=#2\relax} %%problem usinig mu with calc (req by mathtools) loaded
\usepackage{sources}
%\usepackage{length}
%%%%%%%%%%%%%%%%%%%%%%%%%%%%%%%%%%% Funzioni per questo file main
\usepackage{mathOp}
\usepackage{beamersetup}
\def\status{coazione}
\def\keeptrying{coazione}
\usepackage{LocalF}
%%%%%%%%%%%%%%%%%%%%%%%%%%%%%%%%%
\title{Secondo Semestre 20/21}
% A subtitle is optional and this may be deleted
\subtitle{Multi-Wavelenght detectors and their calibration, radiation transfer in ISM, and what ever come on us (Teorie gravitazione, condensed matter e dispositivi fotonici}
\date{Gi\'a cominciano lezioni, \today}
% - Either use conference name or its abbreviation.
% - Not really informative to the audience, more for people (including
%   yourself) who are reading the slides online
% Let's get started
\begin{document}

\addtobeamertemplate{block begin}{\setlength\abovedisplayskip{2pt}\setlength\belowdisplayskip{2pt}\setlength\abovedisplayshortskip{2pt}\setlength\belowdisplayshortskip{2pt}}

\addtobeamertemplate{block begin}{\vspace*{-3pt}}{}
\addtobeamertemplate{block end}{}{\vspace*{-3pt}}

\begin{frame}
  \titlepage
\end{frame}

\begin{frame}{tempo interno per}
\tableofcontents[onlyparts]
\end{frame}

% Section and subsections will appear in the presentation overview
% and table of contents.
%\frame{\tableofcontents[onlyparts]}
%\begin{frame}{Argomenti}
%  \tableofcontents[part=1,hideallsubsections%,pausesections
%  ]
%  % You might wish to add the option [pausesections]
%\end{frame}

\part{Organizzazione}\linkdest{organization}

\begin{frame}{granular: asd luglio - stellare luglio}
    \begin{itemize}
        \item tempo: 10.30-13, 15-19, 23.30-3.30
        \item asd-ripetere: testgof, cramer-rao, likelihood for binned data and extended likelihood
        \item workout-compiti(fare cose che abbiano senso): 30min/exercise exams with sol/45min/per exercise exams without sols
        \item finire rad transf: mihalas:?? (2.00h)
        \item home: bagni, piatti, iola vicino susino per basilico/prezzemolo/??
        \item impianto: campanello (0.30h), camera(1.30h), luce 3 vie(0.30h), luci sala (1.30h), cambio fili sala/soffitta(2.00h), ripetitore wifi in terrazza?
        \item workout: bike(mar 19.30-21, mer 19.30-21)
    \end{itemize}
\end{frame}

\begin{frame}[allowframebreaks]{weekly}
Lentezza, fatica, lontananza della concretezza. 
\begin{itemize}
    \item Lun - 10-11: rev intest; 11-12: d.essa (rosia); 13-14: pranzo/organizzare 14.30-17.30: asd exams; 17.30-19: ripetere testgof; 19-21.30: impianto/orto, 22-23.30:cena 23.30-01: ripetere testgof, 01-04: asd ripetere
    \item Mar - 10.30-12.30: asd ripetere testgof, 12.30: rev parte cowan stime intervallari, formule per cumulante pois, ... 14.00-15.00: pranzo 15.30-18: asd exams exercises, 18.30-19.30: dr.essa, spesa 19.30-21: asd ripetere, 21-22.30: cena 22.30-0.30: asd exams, 0.30-2.30: ripetere testgof
    \item Mer - 11-12.00: telefono meccanico/porto macchina e torno con autobus 12.30.14.30: asd exams, 14.30-15.30: pranzo, 16-19: asd exams, 19-21: bike, 21-22: orto, 22.00-23.00: cena 23-02: mihalas/stellare, 2.00-5.00: asd exams
    \item Gio - 10.30-13.30: ripetere testgof/finire rev CI, 14-15: pranzo 15-16.30: esame asd 22/06 16.30-19: asd ripetere, 19.30-21.30: bike, 22-23: cena, 23-1.30: ripetere testgof, 1.30-4: exams asd
    \item Ven. 10-12: ripetere asd; 12-14: asd exams; 14-15: pranzo; 15-16.30: ripetere asd; 16.30-18: asd exams; 18-19.30: ripetere asd; 19-21: bike; 21-22: cena; 22-24: asd exams; 0-2: asd exams; 2-4: asd exams
    \item Sab - 11-13.30: ripetere asd, 13.30-14.30: coop, 14.30-15.30: pranzo; 15.30-17.30: ripetere asd, 17.30-19.30: asd exams 19.30-21: workout punchbag 21.30-22.30: cena; 22.30-0: asd exams, 0-2: asd exams; 02-04: ripetere asd
    \item Dom - 10-12: sforzo deterlinazione coscienza concretezza esame asd: asd exams, 12-13: ripetere asd, 14-15: pranzo, 15-17: asd exams, 17-19.30: asd exams, 19.30-21: asd ripetere, 21-22:cena 22-00: asd exams, 0-02: asd ripetere
\end{itemize}
\end{frame}

\begin{frame}[label={why}]{Emploi du temp}
\begin{itemize}
    \item Lun. 9-11: condensed matter physics/Plasmi teoria cinetica 14-16:Fisica dei dispositivi fotonici 17-19: Fisica del mezzo diffuso cosmico 18-19: fisica delle onde gravitazionale
    \item Mar. 9-11:Condensed Matter/Astroparticelle 14-16: Fisica delle onde gravitazionali/Teorie delle gravitazioni 16-19: Astrofisica osservativa
    \item Mer. 11-13: Teorie della gravitazione
    \item Gio. 9-11: AstroParticelle/Plasmi teoria cinetica 14-16: Fisica delle onde Gravitazionali 15-16: Teorie della gravitazione 16-19: Astrofisica osservativa
    \item Ven. 9-11: AstroParticelle 15-18: Dispositivi fotonici 17-19: Mezzo Diffuso
\end{itemize}
\end{frame}

\section{AO}

\begin{itemize}
    \item Visibility exercise
    \item star field exercise
    \item spectrum analysis
\end{itemize}

\section{ISM}

\begin{itemize}
    \item How we know there's a ISM? (19/02)
    \item What is a gas? Condition for using kinetic approx? What ''pressure'' mean? What ''moving'' means? (19/02)
    \item ISM absorption: clouds - IGM absorption: Ly forest (26/02)
    \item Curve of Growth (26/02):
    \item Exercise: Curve of Growth for? (26/02)
    \item Linewidth, turbulence and element abundances (01/03)
    \item 
\end{itemize}


\section{asd}

\begin{itemize}
    \item ripetere/registrare
    \item compiti
    \item succo/deeper
\end{itemize}

\subsection{Compiti}

\begin{itemize}
    \item 06/04 - 10-11.30: passaggio a livello (17/09/20), 22-23.30: membrane cellulari (10/02/20)
    \item 07/04 - 10-11.30: generazione 2 numeri casuali indipendenti (21/11/19), 22-23.30: (09/09/19) 1) $x_1, x_2\in U[0,1]$, $x=$\lbt{x_2:\ x_1>f}{mx_2}. 2) Misura massa m con contributo f di nuova fisica
    \item 08/04 - 10-11.30: (08/01/20) $x_1,x_2\in U[-m/2,m/2]$ $y=|x_1+x_2|$ 22-23.30: (01/02/18) 1) Prove di esame a risposta multipla 2) Test per dado truccato
    \item 09/04 - 10-11.30: (16/07/19) generazione numeri pseudo-casuali 22-23.30 (28/05/19) 1) $p(x;\theta)=\frac{\theta3^{\theta}}{x^{\theta+1}}$ 2) probabilit\'a genotipi AA, Aa, aa
    \item 10/04 10-11.30 (11/02/19) 1) Distribuzione esponenziale 2) Contachilometri senza decimale 22-23.30 (15/01/21) Sistema di acquisizione con pre-scale 
    \item 11/04 10-11.30 (08/02/21) Rivelatore vita media di particelle cariche pi\'u rumore
\end{itemize}

\subsection{Ripetere/registrare/succo/deeper}

\begin{itemize}
    \item 
\end{itemize}

\section{ISM}

\subsection{Program}

\begin{itemize}
    \item Physical state of medium (microscale). Heliosphere. ISM (Atomic, Molecular, Dust), IGM (Atomic, Dust): thermal, non-thermal.
    \item Dynamic state. Flows: Large scale (\SIrange{e2}{e3}{\parsec}), Small scale ($<\SI{1}{\parsec}$), kinematics.
    \item Constitution/Structure of the Medium
    \item Energetics Imbalances
\end{itemize}

\begin{itemize}
    \item Exercise: Trans-Ocean cable. Heavyside problem: How quickly can I press the button without having signal smeared-out.
    Assuming: plasma of ISM not optically thin, index of refraction calculation, behaviour (fluid-like (MHD), gas+grain+charges)
    \item 
\end{itemize}

\end{document}