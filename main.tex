\documentclass[10pt,xcolor={usenames},fleqn,mathserif,serif]{beamer}
%%%Usefull link
%tikz-equations:
%http://www.wekaleamstudios.co.uk/posts/creating-a-presentation-with-latex-beamer-equations-and-tikz/
\hypersetup{pdfpagemode=FullScreen}
%% colors
\definecolor{bittersweet}{rgb}{1.0, 0.44, 0.37}
\definecolor{brilliantlavender}{rgb}{0.96, 0.73, 1.0}
\definecolor{antiquefuchsia}{rgb}{0.57, 0.36, 0.51}
\definecolor{violetw}{rgb}{0.93, 0.51, 0.93}
\definecolor{Veronica}{rgb}{0.63, 0.36, 0.94}
\definecolor{atomictangerine}{rgb}{1.0, 0.6, 0.4}
\definecolor{darkgray}{rgb}{0.66, 0.66, 0.66}
\definecolor{brightcerulean}{rgb}{0.11, 0.67, 0.84}
\definecolor{cadmiumorange}{rgb}{0.93, 0.53, 0.18}
\definecolor{ochre}{rgb}{0.8, 0.47, 0.13}
\definecolor{midnightblue}{rgb}{0.1, 0.1, 0.44}
\definecolor{lemon}{rgb}{1.0, 0.97, 0.0}
\definecolor{grey}{rgb}{0.7, 0.75, 0.71}
\definecolor{amber}{rgb}{1.0, 0.75, 0.0}
\definecolor{almond}{rgb}{0.94, 0.87, 0.8}
\definecolor{bf}{RGB}{88, 86, 88}
\definecolor{bb}{RGB}{177, 177, 177}
%%%%%%%%%%%%%%%%%%%%%%%%%%%%%%%%%%% importa pacchetti
\usepackage{usepkg}
%%%%%%%%%%%%%%%%%%%%%%%%%%%%%%%%%%% Funzioni generali
\usepackage{functions}
%http://tex.stackexchange.com/questions/246/when-should-i-use-input-vs-include
\newcommand{\setmuskip}[2]{#1=#2\relax} %%problem usinig mu with calc (req by mathtools) loaded
\usepackage{sources}
%\usepackage{length}
%%%%%%%%%%%%%%%%%%%%%%%%%%%%%%%%%%% Funzioni per questo file main
\usepackage{mathOp}
\usepackage{beamersetup}
\def\status{coazione}
\def\keeptrying{coazione}
\usepackage{LocalF}
%%%%%%%%%%%%%%%%%%%%%%%%%%%%%%%%%
\title{Secondo Semestre 20/21}
% A subtitle is optional and this may be deleted
\subtitle{Multi-Wavelenght detectors and their calibration, radiation transfer in ISM, and what ever come on us (Teorie gravitazione, condensed matter e dispositivi fotonici}
\date{Gi\'a cominciano lezioni, \today}
% - Either use conference name or its abbreviation.
% - Not really informative to the audience, more for people (including
%   yourself) who are reading the slides online
% Let's get started
\begin{document}

\addtobeamertemplate{block begin}{\setlength\abovedisplayskip{2pt}\setlength\belowdisplayskip{2pt}\setlength\abovedisplayshortskip{2pt}\setlength\belowdisplayshortskip{2pt}}

\addtobeamertemplate{block begin}{\vspace*{-3pt}}{}
\addtobeamertemplate{block end}{}{\vspace*{-3pt}}

\begin{frame}
  \titlepage
\end{frame}

\begin{frame}{tempo interno per}
\tableofcontents[onlyparts]
\end{frame}

% Section and subsections will appear in the presentation overview
% and table of contents.
%\frame{\tableofcontents[onlyparts]}
%\begin{frame}{Argomenti}
%  \tableofcontents[part=1,hideallsubsections%,pausesections
%  ]
%  % You might wish to add the option [pausesections]
%\end{frame}

\part{Intro}\linkdest{intro}

\begin{frame}[label={why}]{Why?}

\end{frame}

\begin{wordonframe}{Cosa provo?}

\end{wordonframe}

\part{Steve blowing}
\begin{frame}{Biblio Radiative transfer}
$\S$ Mihalas
– "Stellar Atmospheres" (1970) (?)
– "Stellar Atmospheres" (1978) (*)
– with Hubeny : "Theory of Stellar Atmospheres" (2014)
$\S$ simpler
– Novotny: "Introduction to stellar atmospheres and interiors" (1973)
– Gray: "The observation and analysis of stellar photospheres" (2005) – Bohm-Vitense: "Introduction to stellar astrophysics I, II" (1989)
$\S$ harder
– Cannon: "The transfer of spectral line radiation" (1985)
– Mihalas $\&$ Mihalas: "Foundations of Radiation Hydrodynamics" (1984) (*) – Castor: "Radiation Hydrodynamics" (2004)
$\S$ my stuff
– IART: bachelors-level radiative transfer (1991, 2015, 20??) (*)
– RTSA: masters-level Mihalas popularization (2003, 20??) (*)
– ISSF: bachelors-level introduction to NLTE (1993) (*)
– Monterey: PhD-level refresher NLTE chromospheric lines for IRIS (2012) (*) – Cartagena: tutorial non-E hydrogen diagnostics for ALMA (2016) (*)
\end{frame}

\begin{frame}[allowframebreaks]{Reg Lez (rec)}

\begin{itemize}
\item 4 Ottobre. P cygni spectrum: frobbidden/permitted lines, broadening, shock, autoionization, velocity gradient.
LTE, coupling rules, line ratio (ration intensity members of multiplets),

 map wavelength to pixels, Stati atomici,
\item 8 ottobre. (Processi) armonic oscillator one characteristic frequency-planck function used in E. coefficient  (TE);broaden due to spontaneous decay: $A\propto\Gamma$. Lorentzian profile. Complete redistribution: efficience of absorption do not affect re-emission.
Seeing through optical thin medium sampling different temperature. Rotational profile. Voigt profile $H(\frac{\Delta\omega}{\Delta\omega_D},\frac{\Gamma}{\Delta\omega_D})=H(v,a)$. Equivalent width (line). Regime in cui le ali sono trascurabili: small a. Curve of growth.
\item 9 Ott (Osservativa). Colori: Bande fotometriche Johnson, Str\:omgren. Interferenza: separazione ordine (dispersion). Fotometria: opacit\'a atmosfera vs spazio; osservazione relative da Terra (bootstrapping) (?MK system); plate vs CCD; Vega vs AB-mag. Atmosfera: real/immaginary part of extintion; horizon problem; fluctuation of index of refraction (seeing): lunghezza di coerenza, turbulence. PSF: Gaussian (uncorrelate fluctuation). RA/DEC and field rotation.

\item Lun mart: resonant de-excitation, molecules

\item 18 Ottobre. CCD; low resolution spectra; strumental noise (temerature), ...
Frank-condon factor; molecular reaction
thermal electric fluctuation
\item 29 ottobre. Frank-Condon facor. Electron configuration in molecules. H molecular almost transparent. Radiative transfer; atmosphere: albedo, radiative equilibrium, atmosphere, star wind.
continuum media vs cosmological conditions: 21cm line
\item 30 ottobre. Astrofisica: come si passa da descrizione miscroscopica a macroscopica? (galaxies-fluid + poisson eq); collisional/less: momentum transfer, kinetic (dis)equilibrium, E/L description (c.e.). Coupling between mass, momentum, energy conservation: hierarchy of momenta. Duck Lion. Intro to turbulence. Stress tensor: shear is the non-diagonal term. Shock: discontinuity surface (spatial scale go to zero). continuous conditions across shock: shock adiabat ($\textasciicircum$). Discontinuity: casello dove la densit\'a dopo \'e minore di quella prima. How we can define T after shock front passage. Precursor. Ionization front, Str\"omgren sphere.

\item mon/tue (5-6/11) ???
Interferometry, fossil, HII region shock(?),trinity: self-similar solution time since ignition, dimension what is energy?
Blast solution

\item 8/11. shock front, selfsimilarity, HII regions, molecular cloud, barotropic shock front become corrugate: we can't mantains self-similar solutions; precursor. Problem: $V(x)$, $D(x)$, $\Pi(x)$ functions of $M_0$ (Mach number) at $x=1$.
wind: mechanical luminosity (work done on environment),

\item 12/11. Evaporation, outflow: solution (Bondi-Littleton-Hoyle ('49), Parker ('66)). Eddington limit (AGN/accretion around BH). Diffusion vs wind; lines. Milne instability in stellar wind: Milne '27, Rybyhky Owacky '80s, Lucy white '80s, Lucy Solomon '80s. P Cygni profile: Mihalas and Mihalas '84, Hubeny Mihalas '16

\item 13/11. star cluster simulation: mass (RV 1/f) $->$ IMF $->$ LF(m), distribute star into grid (100x100), sorting algorithm, noise, recover knonws in a field; completeness. PSF. Decompose vs deconvolve. (2 weeks ago: noise inside detector multiplicate: interferometer crruciality)

\item 15/11. radiative acceleration. isotropy-OpticallyThick-dense-EddingtonFactor, Thompson electron opacity; element diffusion: Hg-Mn stars.

Outflow problem: velocity field assumption (we solve radiative transfer not dynamics). 
Jets, de Lavel problem. Buoyancy. Rayleigh-Taylor instability.

\item 19/11. (pavlek seminario: galactic halos, ...) Buoyancy instability: cannot drink chinotto falling from leaning tower, Rayleigh-Taylor instability; shear $->$ vorticity; Kelvin-Heltzmholtz instability; (equilibrium is a instantaneous statement) ''act like viscosity'': turbulence. Mixing length: prandtl vs bierman -vitense; $\alpha\propto H_p$. ''If anything worth bathroom reading something on fluid it did''

\end{itemize}

\end{frame}

\part{Succo}\linkdest{succo}
\section{Spectroscopy}

\section{Radiative trasfer equation}

\begin{frame}{SSP: Synthetic stellar pop}

\end{frame}

\part{Processi Astrofisici}\linkdest{processi}
\section{Radiazione}\linkdest{atomsphotons}
\input{radiation}

\part{Astrofisica osservativa}\linkdest{osservativa}
\section{Rivelatori}
%\input{}

%\part{Basi NMR: nuclei, spin, campi magnetici e magnetizzazione della materia}\label{part:}
%\input{NMR}

\end{document}